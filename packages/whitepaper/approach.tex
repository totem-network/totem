\section{Approach}

The approach takes a look at what techniques can be used
to solve the problems mentioned above and how to combine
them in a easy to use application.

With smart contract wallets
\cite{smartContractWallet}
handling multiple keys,
for instance with one key per device, and implementing
custom recovery solution less difficult. Those wallets
are on the Ethereum blockchain with keys mapped to
the DID Document. A recovery process similar to existing
account systems can be implemented using hardware
security modules.
\cite{recovery}

Those wallets also store the reference hash of the DID
document uploaded to the IPFS network. It enables
resolution of public keys by knowing the wallet
address or a human readable domain.

To prepare data for performance before encryption unique
methods must be used for different data types. Images
as an example can be transformed to different sizes. A blury
low resolution placeholder with a size of 1-2 Kb cached
on the device makes scrubbing through all images quickly
feel instant while the thumbnails for the next page are
loaded in the background before they appear on the users
screen. Only when a user clicks on an image a bigger size
that fits the screen resolution gets requested from the
network.
\cite{images}
This approach increases bandwidth when uploading
the file, but saves a lot of it the more often the
file gets requested.

Federated learning enables artificial intelligence to train
its models on a clients device with the data store locally.
\cite{ai}

To work on all platforms and devices the client can be
developed as a progressive web application. Utilizing the
webs standards makes it accessible and platform agnostic.
\cite{pwa}